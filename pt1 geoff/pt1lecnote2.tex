\chapter[Lec 2: Some examples in calculating probabilities with rolling of dice]{Lec 2: Some examples in \\calculating probabilities \\with rolling of dice}

\section{Is probability easier now than in 1560? (Carden's Counting)}

In this worked example, we investigate the key problem: 
\[\text{\textit{How many ways can we roll three dice such that we obtain a sum of 3?}}\]

This means that the set of three dice rolled either contains 
a three on one of its faces (e.g., the outcome \((3,\ldots ,\ldots )\)), 
or that any combination of the faces sum to three (e.g. \((2,1,x)\)). 

\bigskip 

In order to solve this question, we must employ combinatoric strategies. 

\subsection{Combinatorics}

Some key issues that come into play when trying to solve this question include 
overcounting or missing cases. 

\section{Efron's Dice}
