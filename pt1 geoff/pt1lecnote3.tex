\chapter[Lec 3: Examples using Baye's theorem]{Lec 3: Examples using\\Baye's theorem}


\begin{definition}[Conditional Probability]\label{defn:cond prob}
    \(\mathbb{P}(A|B)\) denotes the \textbf{conditional probability} of \(A\) if \(B\) has occured, 
    for \(A,B \in \mathcal{F}\), and is defined by 
    \begin{equation}\label{eq:cond prob}
        \mathbb{P}(A|B) = \mathbb{P}(A|B) = \frac{\mathbb{P}(A\cap B)}{\mathbb{P}(B)}.
    \end{equation}
\end{definition}

A helpful interpretation of this definition is that in order for \(A\) to occur, if \(B\) is given then the event must 
lie within the intersection of \(A\) and \(B\). Then, we incorporate the information from \(B\) already having occured. 
Or, we consider \(B\) our new `sample space' and are looking for the likelihood that the event lands in the event \(A\). 

\bigskip

Next, we introduce the \textbf{Law of Total Probability}. 

\begin{theorem}[Law of Total Probability]\label{thm:law of TP}
    If \(\{B_1,\ldots,B_n\}\) be a partition of \(\Omega\) with \(\mathbb{P}(B_i)>0\) for all \(i\), then
    \begin{equation}\label{eq:law of TP}
        \mathbb{P}(A)= \sum_{i=1}^n \mathbb{P}(A|B_i)\mathbb{P}(B_i).
    \end{equation}
\end{theorem}

When we combine the Law of Total Probability (\cref{eq:law of TP}) together with the definition of conditional probability (\cref{eq:cond prob}), we obtain the following:

\begin{theorem}[Baye's Theorem]\label{thm:Baye's}
    Let \(\{B_1,\ldots,B_n\}\) be a partition of \(\Omega\) with \(\mathbb{P}(B_i)>0\) for all \(i\) and assume \(\mathbb{P}(A)>0\). Then, 
    \begin{equation}\label{eq:Baye's}
        \mathbb{P}(B_j|A) = \frac{\mathbb{P}(B_j)\mathbb{P}(A|B_j)}{\sum_{i=1}^n \mathbb{P}(B_i)\mathbb{P}(A|B_i)}.
    \end{equation}
\end{theorem}

Now, we consider Baye's Theorem in terms of applications to cancer screenings. 
Here, \(B = \{\text{diagnosis is positive}\}\), \(A_i = \{\text{the }i^{\text{th}} \text{ person has cancer}\}\). 
