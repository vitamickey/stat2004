\chapter[Random experiments and probability models]{Random experiments and \\ probability models}

\section{Random experiments}

The basic notion in probability is that of a 
\textbf{random experiment}: an experiment whose outcome 
cannot be determined in advance, but is nevertheless 
subject to analysis. Some examples of random experiments are: 

\begin{enumerate}
    \item tossing a die, 
    \item counting the number of ducks walking along a lake in a given hour, or
    \item measuring the length of 5 footlong subs purchased from a Subway.
\end{enumerate}

We wish to describe these experiments via a mathematical 
model. This model consists of three building blocks: 
a \textit{sample space}, a set of \textit{events} and 
a \textit{probability}. We will now describe each of these building blocks. 

\section{Sample spaces}

Although we cannot predict the outcome of a random experiment 
with certainty we usually can specify a set of possible outcomes. 
This gives the first ingredient in our model for a random experiment. 

\begin{definition}[Sample space]\label{defn:sample space}
    The \textbf{sample space $\boldsymbol\Omega$} of a random experiment 
    is the set of all possible outcomes of the experiment.
\end{definition}

The following are examples of random experiments with their sample spaces. 

\begin{enumerate}
    \item Casting two die consecutively, \[\Omega = \{(1,1),(1,2), \ldots , (1,6),(2,1),\ldots ,(6,6)\}.\]
    \item The number of ducks that might walk past on a given day, \[\Omega =\{0,1,2,\ldots \} = \mathbb{Z}^{\geq 0}.\]
    \item The length of 5 footlong subs purchased from a Subway, \[\Omega = \{(x_1,\ldots ,x_5)\mid x_i\geq0, i=1,\ldots ,5\}.\]
\end{enumerate}

Note that for modelling purposes it is often easier to take the sample 
space larger than necessary. For example, it is not likely for a footlong 
sub to be 100m long even though the sample space would include this measurement, 
and we would not expect 1000 ducks to walk past a regular lake. 

\section{Events and sigma algebras}

Often we are not interested in a single outcome but in whether 
one of a \textit{group} of outcomes occurs. Such subsets 
of the sample space are called \textbf{events}, denoted by \(A, B, C, \ldots \). 
We say that event \(A\) occurs if the outcome of the experiment is one of the elements in \(A\). 
\medskip
Examples of events are:

\begin{enumerate}
    \item The event that the sum of two dice is 10 or more, \[A = \{(5,5),(5,6),(6,5),(6,6)\}.\]
    \item The event that we see at most a dozen ducks today, \[A = \{0, 1, \ldots , 12\}.\]
    \item The event that a Subway footlong is an acceptable length (in inches), \[A = [11.5,12.5].\]
    \item The event that out of fifty selected people, 5 are left-handed, \[A = \{5\}.\]
\end{enumerate}


Whilst the notion of events is familiar, in order to further extend 
the notion of events, we introduce the abstraction of sets of events, i.e. \textbf{sigma algebras}. 


\begin{definition}[Sigma algebra]\label{defn:sigma algebra}
    A \textbf{sigma algebra \(\mathcal{F}\)} is a collection of 
    subsets of the sample space \(\Omega\) if:
    \begin{enumerate}
        \item \(\mathcal{F}\) is non-empty
        \item \(A\in\mathcal{F}\Rightarrow A^c\in\mathcal{F}\)
        \item \(A_1, A_2,\ldots \in\mathcal{F}\Rightarrow \bigcup_{i=1}^\infty A_i\in\mathcal{F}\)
    \end{enumerate}
\end{definition}

For an experiment we say that event \(A\) \textbf{occurs} if its outcome \(\boldsymbol\omega\)
is an element of \(A\), i.e. \(\omega\in A\). 

\begin{example}[Sigma algebras]
    Some examples of sigma algebras include:
    \begin{enumerate}
        \item If \(\Omega\) is a sample space, the power set (the set of all 
        subsets of a set where \(|\mathcal{P}(\Omega)|=2^{|\Omega|}\))
        is an event space.
        \item \(\{\varnothing, \Omega, \{1, 2\}, \{3, 4\}, \ldots \}\) when you throw a dice multiple times. 
    \end{enumerate}
\end{example}

\begin{exercise}
    Prove these.
\end{exercise}

In the context of sigma algebras, an event is an element of 
a sigma algebra and itself is a set. So, we can apply our usual set operations:

\bigskip
For \(A,B\in\mathcal{F}\),

\begin{enumerate}
    \item the set \(A\cup B\) (\(A\) \textbf{union} \(B\)) is the event that either \(A\) \textit{or} \(B\) occur,
    \item the set \(A\cap B\) (\(A\) \textbf{intersection} \(B\)) is the event that \(A\) \textit{and} \(B\) both occur,
    \item the event \(A^c\) (\(A\) \textbf{complement}) is the event that \(A\) does \textit{not} occur, and
    \item if \(A\subset B\) (\(A\) is a \textbf{subset} of \(B\)) then event \(A\) is said to \textit{imply} event \(B\).
\end{enumerate}

Two events \(A\) and \(B\) which have no outcomes in common, i.e. \(A\cap B=\varnothing\), 
are called \textbf{disjoint} events. 

\medskip

% Some additional properties of (finite) event sets include:

% \begin{enumerate}
%     \item If \(A_1, \ldots , A_n\in\mathcal{F} \Rightarrow \bigcup_{i=1}^n A_i\in\mathcal{F}\)\\
%     \item \(\varnothing, \Omega \in\mathcal{F} \). 
% \end{enumerate}


Sigma algebras are useful as they allow us to investigate 
larger collections of events and their properties in a 
nicely defined context. With them, we can discuss 
events more generally and establish 
rigorous definitions and axioms for probability measures. 

\section{Probability}

\begin{definition}[Probability]\label{defn:probability}
    A probability \(\mathbb{P}\) is a rule (function) which assigns 
    a positive number to each event and satisfies the following axioms:\\
    Axiom 1: \(\mathbb{P}(A)\geq0 \forall A\in\mathcal{F}\)\\
    Axiom 2: \(\mathbb{P}(\Omega)=1\)\\
    Axiom 3: For any sequence \(A_1,A_2,\ldots \) of \textit{disjoint} events we have
    \[\mathbb{P}(\bigcup_{i}A_i) = \sum_i \mathbb{P}(A_i)\tag{1.3}.\]
\end{definition}

\section{Conditional probability and independence}

\subsection{Chain rule}

\begin{theorem}[Baye's Theorem]
    Wow it's Baye's!
\end{theorem}

See \cref{thm:Baye's}

\subsection{Law of total probability and Baye's theorem}


% \begin{example}[Diagnostic testing]\label{eg:diagnostic testing}
    
% \end{example}

\subsection{Independence}

Incomplete. 